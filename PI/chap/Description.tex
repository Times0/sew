\section{Description du projet}

Cette section vise à fournir une vue d'ensemble détaillée du projet de développement du système de gestion des contenants réutilisables. Elle comprendra une explication des objectifs du projet, de sa portée, ainsi que des fonctionnalités et des caractéristiques principales du système à développer.

\subsection{Objectifs du projet}

Les principaux objectifs du projet comprennent :\\

\begin{itemize}
    \item[$\bullet$] Concevoir un système de gestion des contenants réutilisables capable de répondre aux besoins spécifiques de \emph{SEW-Usocome} en matière de logistique inter-sites et de surveillance des points relais.
    \item[$\bullet$] Développer des fonctionnalités de comptage et de catégorisation des contenants réutilisables à l'aide de techniques avancées de machine learning, telles que les réseaux de neurones convolutifs.
    \item[$\bullet$] Intégrer le système avec les systèmes existants de \emph{SEW-Usocome} pour assurer une transition transparente et une compatibilité avec les processus de travail actuels.
    \item[$\bullet$] Fournir une interface conviviale pour les utilisateurs finaux afin de faciliter la gestion et la surveillance des contenants réutilisables.
\end{itemize}

\subsection{Portée du projet}

Le projet couvrira les aspects suivants :\\

\begin{itemize}
    \item[$\bullet$] Développement du logiciel pour le système de gestion des contenants réutilisables, prenant en compte la variété de types de bacs plastiques et caisses métalliques utilisés dans les usines et leur répartition entre les sites.
    \item[$\bullet$] Intégration du système avec les systèmes existants de l'entreprise,\textcolor{red}{ en tenant compte des plans et de la disposition physique des usines} pour assurer une compatibilité optimale.
	\item[$\bullet$] Formation des utilisateurs finaux sur l'utilisation du système, en mettant l'accent sur les caractéristiques spécifiques des bacs et les procédures liées à leur gestion. De plus, le système doit être conçu de manière à ce que son fonctionnement puisse être expliqué de manière claire et concise lors de la présentation à l'entreprise.
	\item[$\bullet$] Support et maintenance du système après son déploiement, en prenant en compte les éventuelles modifications dans les types de bacs ou les plans des usines. L'entreprise doit être en mesure de réaliser des opérations de maintenance courantes et d'ajouter de nouvelles catégories de bacs sans recourir à notre assistance technique, assurant ainsi une autonomie et une flexibilité accrues dans la gestion du système.
\textcolor{red}{\item[$\bullet$] Ajout de fonctionnalités de reporting et d'analyse avancées pour permettre à l'entreprise de surveiller et d'analyser les tendances de gestion des stocks, les performances opérationnelles et d'autres métriques clés.}

\end{itemize}

\subsection{Fonctionnalités principales}

Les fonctionnalités principales du système pour la gestion inter-sites incluront :\\

\begin{itemize}
    \item[$\bullet$] Comptage automatique et précis des contenants réutilisables vides sur rolls à l'entrée et à la sortie des zones de stockage principales, en utilisant des technologies de vision par ordinateur couplées à du machine learning et à du suivi dynamique (vidéo).
    \item[$\bullet$] Comptage automatique et précis des contenants réutilisables vides sur palettes des zones de stockage principales, en utilisant des technologies de vision par ordinateur couplées à du machine learning.
    \item[$\bullet$] Catégorisation intelligente des contenants réutilisables vides en fonction de leur type permettant une gestion fine et efficace des stocks.\\
\end{itemize}

Les fonctionnalités principales du système pour la gestion intra-site incluront :\\

\begin{itemize}
    \item[$\bullet$] Comptage automatique et précis des contenants réutilisables vides sur les zones de stockage relais, en utilisant des technologies de vision par ordinateur couplées à du machine learning grâce à des photos prises régulièrement (caméra sur l'emplacement).\\
\end{itemize}

Les fonctionnalités principales du système général incluront :\\

\begin{itemize}
    \item[$\bullet$] Interface de suivi du stock de bacs vides du site (chiffré pour chaque catégorie) pour déterminer s'il faut prévenir les autres sites pour obtenir de nouveaux bacs, ainsi que le nombre de bacs vides sur chaque relais pour visualiser les lacunes et planifier le réapprovisionnement.
    \item[$\bullet$] Gestion proactive des alertes en cas d'approche de pénurie de contenants vides, avec des fonctionnalités de notification en temps réel pour permettre une réaction rapide et efficace aux événements critiques.\\
\end{itemize}

\subsection{Autres fonctionnalités possibles}

Les fonctionnalités en plus pouvant êtres incluent :\\

\begin{itemize}
    \item[$\bullet$] Comptage automatique et précis des contenants réutilisables pleins sur les zones de stockage relais, en utilisant des technologies de vision par ordinateur couplées à du machine learning grâce à des photos prises régulièrement (caméra sur l'emplacement).
    \item[$\bullet$] Comptage automatique et précis des contenants réutilisables pleins sur les zones de stockage principales.
    \item[$\bullet$] Système d'alerte sur le mauvais placement des bacs (vides ou pleins) aux abords des zones relais en cas de manque de place ou de négligence, pouvant entraîner des gênes ou des dangers.\\
\end{itemize}
