\documentclass[aspectratio = 43]{beamer}
\usetheme{AnnArbor}
% set captions with numbers
\setbeamertemplate{caption}[numbered]


\usepackage{amsmath, nccmath}
\usepackage{multirow}
\usepackage[french]{babel}
\usepackage[latin1]{inputenc}
\usepackage[T1]{fontenc}
\usepackage[backend=biber]{biblatex}
\addbibresource{biblio.bib}


\definecolor{halfgray}{gray}{0.8}

\usecolortheme{whale}
\setbeamercolor{frametitle}{parent=structure,bg=halfgray}

% Include other packages here, before hyperref.
\DeclareGraphicsExtensions{.pdf,.png,.jpg,.jpeg}
\graphicspath{{Figures/}} %Where the figures folder is located

%%%%%%%%%%%%%%%%%%%%%%%%%%%%%%%%%%%%%%%%%%%%%%%%%%
\usepackage{pgffor}
\usepackage{listings}
\usepackage[%font=bf,
skip=3pt]{caption}

\captionsetup[lstlisting]{font={tiny,tt}}
\setbeamerfont{caption}{size=\tiny}
%\setlength\abovecaptionskip{-1pt}
\usepackage{pythonhighlight}
\usepackage{xcolor}

\definecolor{codegreen}{rgb}{0,0.6,0}
\definecolor{codegray}{rgb}{0.5,0.5,0.5}
\definecolor{codepurple}{rgb}{0.58,0,0.82}
\definecolor{backcolour}{rgb}{0.95,0.95,0.92}

\lstdefinestyle{mystyle}{%
    backgroundcolor=\color{backcolour},
    commentstyle=\color{codegreen},
    keywordstyle=\color{magenta},
    numberstyle=\tiny\color{codegray},
    stringstyle=\color{codepurple},
    basicstyle=\ttfamily\footnotesize,
    breakatwhitespace=false,
    breaklines=true,
    captionpos=b,
    keepspaces=true,
    numbers=left,
    numbersep=1pt,
    showspaces=false,
    showstringspaces=false,
    showtabs=false,
    tabsize=2,
}

\lstset{style=mystyle}
%\usepackage{python}
%%%%%%%%%%%%%%%%%%%%%%%%%%%%%%%%%%%%%%%%%%%%%%%%%%

\title{Revue de Projet}
\subtitle{Classification et Comptage de contenants vides}
\author[]{\tiny Thomas CHECCHIN $-$ Dorian CHEVALERIAS $-$ Nicolas TO VAN
  TRANG $-$ Za�d GHALI}
\institute[]{\textbf{T�l�com Physique Strasbourg}\\
\textbf{SEW - Usocome}}
\date{\tiny \today}


\begin{document}

%%%%--1--%%%%
\begin{frame}
  \begin{minipage}[t][0.2\textheight][t]{\textwidth}
    \centering
    \includegraphics[scale=0.3,width=0.2\textwidth]{tps-logo.png}
    \label{fig:tps_logo}
    \hspace*{4cm}
    \includegraphics[scale=0.3,width=0.2\textwidth]{SEW_logo.jpg}
    \label{fig:sew_logo}
  \end{minipage}
  \titlepage%
\end{frame}
%%%% --2--%%%%
\begin{frame}
  \frametitle{Plan}
  \tableofcontents
\end{frame}
%
% %%%% --3--%%%%
% \begin{itemize}
% \item Pr�sentation entreprise
% \item Etat actuel
% \end{itemize}

\section{Remise en contexte}
\subsection{Pr�sentation de l'entreprise}
\begin{frame}
  \frametitle{Pr�sentation de l'entreprise}
  \textbf{SEW Usocome}

  \begin{minipage}{.5\textwidth}
  \begin{itemize}
  \item filiale fran�aise du groupe allemand \textbf{SEW-EURODRIVE}
  \item usines � Haguenau, Brumath et Forbach
  \item propose des solutions d'automatisme pour des applications de
    mouvement (moteur �lectrique, servomoteur..)
  \end{itemize}
\end{minipage}%
\begin{minipage}{.5\textwidth}
  \begin{figure}
    \centering
    \includegraphics[scale=1,width=0.9\textwidth]{sew_haguenau.jpg}
    \caption{Vue de l'usine de Haguenau}
    \label{fig:sew_haguenau}
  \end{figure}
\end{minipage}
\end{frame}
%
% %%%% --4--%%%%
\subsection{Etat actuel}
\begin{frame}
  \frametitle{Etat actuel}

  Gestion des stocks inexistantes entre les zones de production et de
  stockage et les autres usines.

  \textbf{Probl�matique~:} Avoir constamment des contenants vides sur
  les zones de production et suffisamment de contenants utiles � la
  production sur chaque site

  \begin{minipage}{.5\textwidth}
    \begin{figure}
      \centering
      \includegraphics[width=0.7\textwidth, height=0.47\textheight]{sew_production.jpg}
      \caption{Image d'une zone de production d'Haguenau}
      \label{fig:sew_production}
    \end{figure}
  \end{minipage}%
  \begin{minipage}{.5\textwidth}
    \begin{figure}
      \centering
      \includegraphics[scale=1,width=0.9\textwidth]{sew_entrepot.jpg}
      \caption{Image de la zone de stockage de Haguenau}
      \label{fig:sew_entrepot}
    \end{figure}
  \end{minipage}
\end{frame}%
% %%% --5--%%%
\subsection{Objectifs du projet}
\begin{frame}
  \frametitle{Objectifs du projet}
  \textbf{Objectif principal~:} Classification et comptage des
  contenants vides

  \begin{center}
    \scriptsize
    \begin{tabular}{|p{3cm}|p{4cm}|p{3cm}|}
      \hline
      Objectifs & Crit�res & Moyens \\
      \hline
      Acquisition de donn�es num�riques & 1. Avoir assez d'images des
                                          diff�rentes bo�tes r�parties homog�nement & Utilisation de cam�ras IP \\
                & 2. Acquisition dans nos locaux et � SEW & \\
      \hline
      Classification des types de bo�tes par apprentissage automatique &
                                                                Classification
                                                                des images avec
                                                                environ
                                                                90 \% de pr�cision & Utiliser Python et YOLO \\
      \hline
      Classification des bo�tes vides ou non par apprentissage automatique
                & D�terminer si les bo�tes sont vides ou non avec une
                  pr�cision tout autant �lev�e 90 \% & Utiliser Python
                                                       et YOLO\\
      \hline
      Compter le nombre de bo�tes & 1. Compter celles vides par types
                                    pr�c�demment identifi�es &\\
                & 2. Prendre en compte les bo�tes cach�es &\\
      \hline
      Utilisation et explicabilit� du logiciel & 1. Rendre le logiciel
                                                 utilisable et compr�hensible & Installation
                                                                                finale et contr�le du logiciel\\
                & 2. Possibilit� de changer les types de bo�tes & \\ %rajout de bo�tes par exemple
      \hline
    \end{tabular}
  \end{center}
\end{frame}
%%


% %%%% --4--%%%%
\section{Pr�vision des t�ches � r�aliser}
\begin{frame}
  \frametitle{Pr�vision des t�ches � r�aliser}
  \begin{itemize}
  \item Pr�sentation du cahier des charges + type de cam�ra envisag�
    donc budgets ?
  \item Diagramme de Gantt sur l'enti�ret� du projet
  \item Diagramme de Gantt pr�visionnel avant le R1 et diagramme effective
  \end{itemize}
\end{frame}
%
% %%%% --5--%%%%
\section{Etat de l'art}
\begin{frame}
  \frametitle{Remise en contexte}
  \begin{itemize}
  \item rfid (trop cher)
  \item vision (contours des bo�tes)
  \item Plus important : YOLO mais d'autres existent mais moins
    utilis�s comme \textit{SDD ?}
  \end{itemize}
\end{frame}
%
% %%%% --6--%%%%
\section{Pistes de solutions}
\begin{frame}
  \frametitle{Pistes de solutions}
  \begin{itemize}
  \item Pour les zones de production : comptage apr�s identification
    des bo�tes vide ou non en ayant ou non d�finit quel type de bo�te
  \item Pour la zone de stockage : utilisation des zones strat�giques
    de passages pour une meilleure classification et s'affranchir des
    risques d'identification des bo�tes cach�es
  \end{itemize}
\end{frame}
%
% %%%% --7--%%%%
%\section{Conclusion}
\begin{frame}
  \frametitle{Conclusion}
  Poursuite du PI ?
\end{frame}
% %%%% --7--%%%%
\begin{frame}
  \frametitle{Bibliographie}
  Pour print la biblio il faut utiliser les refs comme ceci
  \citeauthor{bib:ziou}. Pour actualiser la biblio~:
  \begin{itemize}
  \item pdflatex presentation\_R1.tex
  \item biber presentation\_R1
  \item pdflatex presentation\_R1.tex
  \end{itemize}

  \printbibliography
  Faire un bibtex avec au moins~:
  \begin{itemize}
  \item SEW usocome site \textit{usocome.com}
  \end{itemize}
\end{frame}
\end{document}
%
