\section{Contraintes et limites}

Le projet est soumis aux contraintes et limites suivantes :\\

\begin{itemize}
    \item[$\bullet$] Contraintes techniques : Le système doit être compatible avec l'infrastructure informatique existante de l'entreprise et respecter les normes de sécurité et de confidentialité des données.
    \item[$\bullet$] Contraintes financières : Le budget alloué au projet est limité \textcolor{red}{on pourrai ajouter des montants...} et doit être géré de manière efficace pour couvrir tous les coûts de développement, d'implémentation et de maintenance.
	\item[$\bullet$] Contraintes de scalabilité : Le système doit être scalable pour pouvoir gérer efficacement une augmentation du volume de contenants, de sites et de zones relais à mesure que l'entreprise se développe. De plus, le système doit être conçu de manière à faciliter la maintenance et l'évolution future, en permettant l'ajout de nouvelles fonctionnalités et l'adaptation aux changements des besoins de l'entreprise.
    \item[$\bullet$] Contraintes temporelles : Le projet doit être réalisé dans un délai d'une année civile.
    \item[$\bullet$] Contraintes réglementaires : Le système doit être conforme aux réglementations en vigueur en matière de protection des données, de sécurité informatique et d'hygiène et sécurité au travail.
    \item[$\bullet$] Limites de portée : Le projet se concentrera sur la gestion des contenants réutilisables à l'intérieur des sites de l'entreprise et ne couvrira pas la gestion des stocks externes ou des fournisseurs tiers.
    \textcolor{red}{\item[$\bullet$] Contrainte spéciale : Le système doit comprendre l'utilisation de technique de machine learning.}\\
\end{itemize}


