\section{Besoins et objectifs}

Le but du projet est de répondre aux fonctionnalités précédemment
décrites, \textit{i.e.} classifier et compter le nombre de comptenants
vides dans des zones de production et de stockage dans les entrepôts
de l'entrerpise. Pour y parvenir, des objectifs précis ont été
identifiés auxquels doivent répondre notre projet.

Des besoins spécifiques ont identifiés par \textbf{SEW-Usocome}~:\\

\begin{itemize}
\item[$\bullet$] Automatiser le processus de comptage et de suivi des
  contenants réutilisables pour améliorer l'efficacité opérationnelle
  de la logistique.
\item[$\bullet$] Réduire les coûts liés à la gestion manuelle des
  stocks de contenants réutilisables et minimiser les risques
  d'erreurs humaines.
\item[$\bullet$] Optimiser l'utilisation des contenants réutilisables
  en assurant un réapprovisionnement rapide et précis des zones relais
  et des sites de production.\\
\end{itemize}%

De plus, nous avons défini des objectifs précis qui sont les suivants~:\\

\begin{itemize}
 \item[$\bullet$] Connaître l'état de l'art actuel et trouver une
caméra répondant aux besoins de SEW afin de développer rapidement un
modèle fonctionnel,
\item[$\bullet$] Faire l'acquisition de données numériques dans un
premier temps d'images puis potentiellement de vidéo dans l'optique de
réaliser le prochain objectif,
\item[$\bullet$] Réaliser des apprentissages par classification pour
entraîner un modèle à reconnaître les différents types de boîtes et
d'identifier si elles sont vides,
\item[$\bullet$] Concevoir une interface conviviale permettant de
visualiser en temps réel l'état des stocks de contenants réutilisables
et leur répartition sur les zones relais et les sites de
production. Le logiciel doit intégrer le modèle de l'intelligence
artificielle créé, compréhensible pour l'utilisateur et augmentable
\textit{i.e.} où l'ajout de nouvelles boîtes seraient possible.
\end{itemize}
% Les objectifs spécifiques du projet sont les suivants~:\\

% \begin{itemize}
% \item[$\bullet$] Développer un système de comptage automatique et
%   précis des contenants réutilisables en utilisant des technologies
%   avancées telles que la vision par ordinateur et le machine learning.
% \item[$\bullet$] Concevoir une interface conviviale permettant de
%   visualiser en temps réel l'état des stocks de contenants
%   réutilisables et leur répartition sur les zones relais et les sites
%   de production.
% \item[$\bullet$] Mettre en place un système d'alerte pour signaler les
%   situations critiques telles que les pénuries de contenants
%   réutilisables ou les mauvais emplacements des bacs.
% \item[$\bullet$] Intégrer le système avec les systèmes existants de
%   l'entreprise pour assurer une compatibilité et une cohérence des
%   données.
% \end{itemize}
