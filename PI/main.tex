\documentclass{article}
\usepackage[utf8]{inputenc}
\usepackage[T1]{fontenc}
\usepackage[french]{babel}
\usepackage{graphicx}
\usepackage{hyperref}
\usepackage{tocloft}
\usepackage{xcolor}

\title{Cahier des Charges}
\author{Votre Nom}
\date{\today}

\begin{document}

\maketitle

\begin{center}
    \LARGE{Système de Comptage et de Catégorisation des Contenants Réutilisables par Apprentissage Automatique dans la Logistique Industrielle}
\end{center}

\textcolor{red}{En rouge les points a modifier / revoir. Je n'ai pas mis d'image ou de graphe... mais on pourrai en mettre si besoin.}

\tableofcontents
\newpage


\section{Introduction}

\subsection{Objectif du document}

Le présent document vise à définir les exigences et les spécifications
pour le développement d'un système de gestion innovant des contenants
réutilisables de l'entreprise \emph{SEW-Usocome}. Ce système sera
conçu pour répondre aux besoins spécifiques de l'entreprise en matière
de gestion efficace des bacs plastiques et des caisses métalliques
utilisés pour le stockage et le transport des pièces usinées dans ses
usines.

\subsection{Présentation de l'entreprise}

\textbf{SEW-Usocome} est la branche française de l'entreprise
familiale allemande \textbf{SEW-Eurodrive}. Elle est spécialisée dans
la fabrication de systèmes d'entraînement et d'automatisation. En
France, elle possède plusieurs entrepôts, dont la principale à
Haguenau correspond aussi au siège sociale. Les deux autres sont
situés à Bromath et à Forbach. Ces entrepôts servent à stocker les
pièces détachées, les composants et les produits finis destinés à être
distribués aux différents clients et à assurer un approvisionnement
efficace des différents sites de production et des centres de
distribution.

\subsection{Contexte}

La logistique de gestion des contenants réutilisables constitue un
défi majeur dans le domaine industriel, en particulier pour les
entreprises opérant sur plusieurs sites de production et
d'assemblage. La répartition optimale des bacs plastiques entre ces
sites, ainsi que la surveillance des points relais disséminés dans les
usines, sont des éléments clés pour garantir une efficacité
opérationnelle maximale tout en minimisant les coûts.


Dans ce contexte, \emph{SEW-Usocome} se trouve confrontée à des défis
logistiques spécifiques, notamment la gestion du stock global de bacs
entre ses sites de production et d'assemblage, ainsi que la
surveillance des points relais appelées "gares" pour assurer un
approvisionnement constant en bacs vides et éviter toute obstruction
causée par les caisses pleines.


Les équipes logistiques de l'entreprise doivent actuellement gérer un
volume de plus de 100 000 bacs et caisses, comprenant entre 6 et 8
catégories différentes. Ces contenants circulent entre trois sites de
production et d'assemblage, transportés par camions. Cette gestion
nécessite une organisation efficace pour assurer le flux continu de
ces contenants essentiels à la chaîne de production


À l'heure actuelle, la logistique de l'entreprise fonctionne
principalement grâce à des processus manuels et à des systèmes de
suivi rudimentaires pour la gestion des bacs et des caisses
réutilisables. Les équipes logistiques sont chargées d'effectuer des
comptages réguliers et des vérifications physiques pour garantir que
les stocks de contenants sont suffisants et que leur répartition entre
les différents sites est équilibrée.


Ce mode de fonctionnement offre une solution modérée pour la gestion
des stocks, mais il est occasionnellement confronté à des problèmes de
"famine" de contenants, entraînant des situations d'urgence
nécessitant un réapprovisionnement immédiat. Ces interruptions peuvent
avoir un impact négatif sur les opérations quotidiennes de l'usine,
entraînant des retards de production.

\subsection{Définitions}

Avant de procéder à la description détaillée du projet, quelques termes clés méritent d'être définis pour une meilleure compréhension~:\\

\begin{itemize}
\item[$\bullet$] \textbf{Contenants réutilisables :} Il s'agit des
  bacs plastiques et des caisses métalliques utilisés pour le stockage
  et le transport des pièces usinées dans l'usine.
\item[$\bullet$] \textbf{Points relais / gares :} Des emplacements
  disséminés dans les usines où les contenants réutilisables sont
  stockés temporairement pour un accès facile principalement proche
  des machines de production, d'assemblage de pièces.
\item[$\bullet$] \textbf{Roll :} Un chariot sur quatre roulettes
  permettant d'empiler un type de contenant. Dans l'entreprise, chaque
  roll contient un seul type de contenants. En fonction des types, le
  nombre est fixe.
\item[$\bullet$] \textbf{Gestion logistique :} L'ensemble des
  activités visant à planifier, mettre en œuvre et contrôler le flux
  de produits et d'informations tout au long de la chaîne
  d'approvisionnement.
\end{itemize}


\section{Description du projet}

Cette section vise à fournir une vue d'ensemble détaillée du projet de
développement du système de gestion des contenants réutilisables. Elle
comprendra une explication des objectifs du projet, de sa portée,
ainsi que des fonctionnalités et des caractéristiques principales du
système à développer.

\subsection{Objectifs du projet}

Les principaux objectifs du projet comprennent~:\\

\begin{itemize}
\item[$\bullet$] Concevoir un système de gestion des contenants
  réutilisables capable de répondre aux besoins spécifiques de
  \emph{SEW-Usocome} en matière de logistique inter-sites et de
  surveillance des points relais.
\item[$\bullet$] Développer des fonctionnalités de comptage et de
  catégorisation des contenants réutilisables à l'aide de techniques
  avancées de machine learning, telles que les réseaux de neurones
  convolutifs.
% \item[$\bullet$] Intégrer le système avec les systèmes existants de
%   \emph{SEW-Usocome} pour assurer une transition transparente et une
%   compatibilité avec les processus de travail actuels.
\item[$\bullet$] Fournir une interface conviviale et dont les
  donctionnalités sont compréhensibles pour les utilisateurs finaux
  afin de faciliter la gestion et la surveillance des contenants
  réutilisables.
\end{itemize}

% \subsection{Portée du projet}

% Le projet couvrira les aspects suivants~:\\

% \begin{itemize}
% \item[$\bullet$] Développement du logiciel pour le système de gestion
%   des contenants réutilisables, prenant en compte la variété de types
%   de bacs plastiques et caisses métalliques utilisés dans les usines
%   et leur répartition entre les sites.
% \item[$\bullet$] Intégration du système avec les systèmes existants de
%   l'entreprise, en tenant compte des plans et de la
%     disposition physique des usines pour assurer une compatibilité
%   optimale.
% \item[$\bullet$] Formation des utilisateurs finaux sur l'utilisation
%   du système, en mettant l'accent sur les caractéristiques spécifiques
%   des bacs et les procédures liées à leur gestion. De plus, le système
%   doit être conçu de manière à ce que son fonctionnement puisse être
%   expliqué de manière claire et concise lors de la présentation à
%   l'entreprise.
% \item[$\bullet$] Support et maintenance du système après son
%   déploiement, en prenant en compte les éventuelles modifications dans
%   les types de bacs ou les plans des usines. L'entreprise doit être en
%   mesure de réaliser des opérations de maintenance courantes et
%   d'ajouter de nouvelles catégories de bacs sans recourir à notre
%   assistance technique, assurant ainsi une autonomie et une
%   flexibilité accrues dans la gestion du système.
%   \textcolor{red}{\item[$\bullet$] Ajout de fonctionnalités de
%     reporting et d'analyse avancées pour permettre à l'entreprise de
%     surveiller et d'analyser les tendances de gestion des stocks, les
%     performances opérationnelles et d'autres métriques clés.}

% \end{itemize}

\subsection{Fonctionnalités principales}

% Je trouve ça pas très utile est surtout ça nous complexifie beaucoup
% la tâche l'inter sites. Du coup c'est pas non plus super bien
% combiné. Il vaut mieux pas trop en mettre car sinon on sera pénalisé
% je pense
% Les fonctionnalités principales du système pour la gestion inter-sites incluront~:\\

% \begin{itemize}
% \item[$\bullet$] Comptage automatique et précis des contenants
%   réutilisables vides sur rolls à l'entrée et à la sortie des zones de
%   stockage principales, en utilisant des technologies de vision par
%   ordinateur couplées à du machine learning et à du suivi dynamique
%   (vidéo).
% \item[$\bullet$] Comptage automatique et précis des contenants
%   réutilisables vides sur palettes des zones de stockage principales,
%   en utilisant des technologies de vision par ordinateur couplées à du
%   machine learning.
% \item[$\bullet$] Catégorisation intelligente des contenants
%   réutilisables vides en fonction de leur type permettant une gestion
%   fine et efficace des stocks.
% \end{itemize}

Les fonctionnalités principales du système pour la gestion intra-site
doivent inclure le comptage automatique et précis des contenants
réutilisables vides sur les zones de production, \textit{i.e.} les points
relais, en utilisant des technologies de vision par ordinateur
couplées à de l'apprentissage automatique pour la classification grâce
à des photos prises régulièrement (caméra sur l'emplacement).

De manière équivalente, pour les zones de stockage, il sera possible
d'implémenter soit un système similaire au précédent, soit effectuer
un comptage automatique des contenants réutilisables vides sur rolls à
l'entrée et à la sortie de ces zones à des emplacement stratégiquement
définis. Cette fois-ci l'apprentissage se fera sur des vidéos de train
acheminant les contenants aux zones.\\

Ainsi, ces fonctionnalités doivent être intégrées dans le logiciel final
qui comprendra les fonctionnalités principales générales suivantes~:\\

\begin{itemize}
\item[$\bullet$] Interface de suivi du stock de bacs vides du site
  (chiffré pour chaque catégorie) pour déterminer s'il faut prévenir
  les autres sites pour obtenir de nouveaux bacs, ainsi que le nombre
  de bacs vides sur chaque relais pour visualiser les lacunes et
  planifier le réapprovisionnement.
\item[$\bullet$] Gestion proactive des alertes en cas d'approche de
  pénurie de contenants vides, avec des fonctionnalités de
  notification en temps réel pour permettre une réaction rapide et
  efficace aux événements critiques.
\end{itemize}

D'autres fonctionnalités pourraient être rajouté au logiciel comme un
système d'alerte sur le mauvais placement des bacs (vides ou pleins)
aux abords des zones relais en cas de manque de place ou de
négligence, pouvant entraîner des gênes ou des dangers.


\section{Besoins et objectifs}

Le projet vise à répondre aux besoins suivants identifiés par \emph{SEW-Usocome} :\\

\begin{itemize}
    \item[$\bullet$] Automatiser le processus de comptage et de suivi des contenants réutilisables pour améliorer l'efficacité opérationnelle de la logistique.
    \item[$\bullet$] Réduire les coûts liés à la gestion manuelle des stocks de contenants réutilisables et minimiser les risques d'erreurs humaines.
    \item[$\bullet$] Optimiser l'utilisation des contenants réutilisables en assurant un réapprovisionnement rapide et précis des zones relais et des sites de production.\\
\end{itemize}

Les objectifs spécifiques du projet sont les suivants :\\

\begin{itemize}
    \item[$\bullet$] Développer un système de comptage automatique et précis des contenants réutilisables en utilisant des technologies avancées telles que la vision par ordinateur et le machine learning.
    \item[$\bullet$] Concevoir une interface conviviale permettant de visualiser en temps réel l'état des stocks de contenants réutilisables et leur répartition sur les zones relais et les sites de production.
    \item[$\bullet$] Mettre en place un système d'alerte pour signaler les situations critiques telles que les pénuries de contenants réutilisables ou les mauvais emplacements des bacs.
    \item[$\bullet$] Intégrer le système avec les systèmes existants de l'entreprise pour assurer une compatibilité et une cohérence des données.\\
\end{itemize}


\section{Contraintes et limites}

Le projet est, néanmoins, soumis aux contraintes et limites suivantes~:\\

\begin{itemize}
\item[$\bullet$] Contraintes financières : Le budget alloué est
  évidemment limité. La solution proposée doit prendre en compte le
  coût de plusieurs caméras qui rentre dans les contraintes de
  l'entreprise. \textcolor{red}{lesquelles caméras utilisées ???}
\item[$\bullet$] Une contrainte d'échelle vient s'ajouter. Le système
  doit être scalable pour pouvoir gérer efficacement une augmentation
  du volume de contenants, de sites et de zones relais à mesure que
  l'entreprise se développe. De plus, l'entrepôt de Haguenau va
  doubler de superficie dans les prochaines années. Le nombre de
  boîtes va donc suivre cette tendance.
\item[$\bullet$] Le système doit être conçu de manière à faciliter la
  maintenance et l'évolution future, en permettant l'ajout de nouveaux
  contenants ou de nouvelles fonctionnalités et l'adaptation aux
  changements des besoins de l'entreprise.
\item[$\bullet$] Le projet doit être réalisé dans un délai d'une année
  débutant en janvier $2024$ et se finissant avant début février $2025$.
\item[$\bullet$] Contrainte technique : Le système doit comprendre
  l'utilisation d'intelligence artificielle.
\item[$\bullet$] Contraintes techniques : Le système doit être
  compatible avec l'infrastructure informatique existante de
  l'entreprise et respecter les normes de sécurité et de
  confidentialité des données.
\end{itemize}


\section{concurrents et état de l'art}
\textcolor{red}{A voir ou vraiment placer cette partie, mais parler de "concurrents" est essentiel. Pour concurrents, parler des solution de services offerts sur le marché (RFID) leur avantage et contraintes (prix...). Et peut être mettre notre état de l'art ici aussi.}

\section{Fonctionnalités attendues}

\textcolor{red}{Il y déjà  une partie fonctionnalités mais on peut peut être ajouter des spécifications plus technique (implémentations...) ici}

\section{Critères d'acceptation}

\textcolor{red}{Si on ne met pas de budget dans la partie limite, on peut en mettre un ici. Sinon on peut aussi mettre des détails sur l'attente de SEW si on en a.}

\section{Exigences techniques}

\textcolor{red}{C'est un peu comme Fonctionnalités attendues mais juste les techniques utilisées}

\section{Planification}

\textcolor{red}{On pourra mettre une roadmap, Gantt, ...}

\section{Ressources}

\textcolor{red}{Je ne sais pas si c'est utile}

\section{Communication et suivi}
\textcolor{red}{Un clin d'oeil à Lampert} 

\section{Annexes}

\section{Conclusion}

\end{document}
